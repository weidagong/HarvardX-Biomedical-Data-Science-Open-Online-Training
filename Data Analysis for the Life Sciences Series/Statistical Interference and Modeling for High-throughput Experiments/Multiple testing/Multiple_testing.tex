\documentclass[]{article}
\usepackage{lmodern}
\usepackage{amssymb,amsmath}
\usepackage{ifxetex,ifluatex}
\usepackage{fixltx2e} % provides \textsubscript
\ifnum 0\ifxetex 1\fi\ifluatex 1\fi=0 % if pdftex
  \usepackage[T1]{fontenc}
  \usepackage[utf8]{inputenc}
\else % if luatex or xelatex
  \ifxetex
    \usepackage{mathspec}
  \else
    \usepackage{fontspec}
  \fi
  \defaultfontfeatures{Ligatures=TeX,Scale=MatchLowercase}
\fi
% use upquote if available, for straight quotes in verbatim environments
\IfFileExists{upquote.sty}{\usepackage{upquote}}{}
% use microtype if available
\IfFileExists{microtype.sty}{%
\usepackage{microtype}
\UseMicrotypeSet[protrusion]{basicmath} % disable protrusion for tt fonts
}{}
\usepackage[margin=1in]{geometry}
\usepackage{hyperref}
\hypersetup{unicode=true,
            pdftitle={Multiple testing},
            pdfauthor={Weida Gong},
            pdfborder={0 0 0},
            breaklinks=true}
\urlstyle{same}  % don't use monospace font for urls
\usepackage{color}
\usepackage{fancyvrb}
\newcommand{\VerbBar}{|}
\newcommand{\VERB}{\Verb[commandchars=\\\{\}]}
\DefineVerbatimEnvironment{Highlighting}{Verbatim}{commandchars=\\\{\}}
% Add ',fontsize=\small' for more characters per line
\usepackage{framed}
\definecolor{shadecolor}{RGB}{248,248,248}
\newenvironment{Shaded}{\begin{snugshade}}{\end{snugshade}}
\newcommand{\AlertTok}[1]{\textcolor[rgb]{0.94,0.16,0.16}{#1}}
\newcommand{\AnnotationTok}[1]{\textcolor[rgb]{0.56,0.35,0.01}{\textbf{\textit{#1}}}}
\newcommand{\AttributeTok}[1]{\textcolor[rgb]{0.77,0.63,0.00}{#1}}
\newcommand{\BaseNTok}[1]{\textcolor[rgb]{0.00,0.00,0.81}{#1}}
\newcommand{\BuiltInTok}[1]{#1}
\newcommand{\CharTok}[1]{\textcolor[rgb]{0.31,0.60,0.02}{#1}}
\newcommand{\CommentTok}[1]{\textcolor[rgb]{0.56,0.35,0.01}{\textit{#1}}}
\newcommand{\CommentVarTok}[1]{\textcolor[rgb]{0.56,0.35,0.01}{\textbf{\textit{#1}}}}
\newcommand{\ConstantTok}[1]{\textcolor[rgb]{0.00,0.00,0.00}{#1}}
\newcommand{\ControlFlowTok}[1]{\textcolor[rgb]{0.13,0.29,0.53}{\textbf{#1}}}
\newcommand{\DataTypeTok}[1]{\textcolor[rgb]{0.13,0.29,0.53}{#1}}
\newcommand{\DecValTok}[1]{\textcolor[rgb]{0.00,0.00,0.81}{#1}}
\newcommand{\DocumentationTok}[1]{\textcolor[rgb]{0.56,0.35,0.01}{\textbf{\textit{#1}}}}
\newcommand{\ErrorTok}[1]{\textcolor[rgb]{0.64,0.00,0.00}{\textbf{#1}}}
\newcommand{\ExtensionTok}[1]{#1}
\newcommand{\FloatTok}[1]{\textcolor[rgb]{0.00,0.00,0.81}{#1}}
\newcommand{\FunctionTok}[1]{\textcolor[rgb]{0.00,0.00,0.00}{#1}}
\newcommand{\ImportTok}[1]{#1}
\newcommand{\InformationTok}[1]{\textcolor[rgb]{0.56,0.35,0.01}{\textbf{\textit{#1}}}}
\newcommand{\KeywordTok}[1]{\textcolor[rgb]{0.13,0.29,0.53}{\textbf{#1}}}
\newcommand{\NormalTok}[1]{#1}
\newcommand{\OperatorTok}[1]{\textcolor[rgb]{0.81,0.36,0.00}{\textbf{#1}}}
\newcommand{\OtherTok}[1]{\textcolor[rgb]{0.56,0.35,0.01}{#1}}
\newcommand{\PreprocessorTok}[1]{\textcolor[rgb]{0.56,0.35,0.01}{\textit{#1}}}
\newcommand{\RegionMarkerTok}[1]{#1}
\newcommand{\SpecialCharTok}[1]{\textcolor[rgb]{0.00,0.00,0.00}{#1}}
\newcommand{\SpecialStringTok}[1]{\textcolor[rgb]{0.31,0.60,0.02}{#1}}
\newcommand{\StringTok}[1]{\textcolor[rgb]{0.31,0.60,0.02}{#1}}
\newcommand{\VariableTok}[1]{\textcolor[rgb]{0.00,0.00,0.00}{#1}}
\newcommand{\VerbatimStringTok}[1]{\textcolor[rgb]{0.31,0.60,0.02}{#1}}
\newcommand{\WarningTok}[1]{\textcolor[rgb]{0.56,0.35,0.01}{\textbf{\textit{#1}}}}
\usepackage{graphicx,grffile}
\makeatletter
\def\maxwidth{\ifdim\Gin@nat@width>\linewidth\linewidth\else\Gin@nat@width\fi}
\def\maxheight{\ifdim\Gin@nat@height>\textheight\textheight\else\Gin@nat@height\fi}
\makeatother
% Scale images if necessary, so that they will not overflow the page
% margins by default, and it is still possible to overwrite the defaults
% using explicit options in \includegraphics[width, height, ...]{}
\setkeys{Gin}{width=\maxwidth,height=\maxheight,keepaspectratio}
\IfFileExists{parskip.sty}{%
\usepackage{parskip}
}{% else
\setlength{\parindent}{0pt}
\setlength{\parskip}{6pt plus 2pt minus 1pt}
}
\setlength{\emergencystretch}{3em}  % prevent overfull lines
\providecommand{\tightlist}{%
  \setlength{\itemsep}{0pt}\setlength{\parskip}{0pt}}
\setcounter{secnumdepth}{0}
% Redefines (sub)paragraphs to behave more like sections
\ifx\paragraph\undefined\else
\let\oldparagraph\paragraph
\renewcommand{\paragraph}[1]{\oldparagraph{#1}\mbox{}}
\fi
\ifx\subparagraph\undefined\else
\let\oldsubparagraph\subparagraph
\renewcommand{\subparagraph}[1]{\oldsubparagraph{#1}\mbox{}}
\fi

%%% Use protect on footnotes to avoid problems with footnotes in titles
\let\rmarkdownfootnote\footnote%
\def\footnote{\protect\rmarkdownfootnote}

%%% Change title format to be more compact
\usepackage{titling}

% Create subtitle command for use in maketitle
\newcommand{\subtitle}[1]{
  \posttitle{
    \begin{center}\large#1\end{center}
    }
}

\setlength{\droptitle}{-2em}

  \title{Multiple testing}
    \pretitle{\vspace{\droptitle}\centering\huge}
  \posttitle{\par}
    \author{Weida Gong}
    \preauthor{\centering\large\emph}
  \postauthor{\par}
      \predate{\centering\large\emph}
  \postdate{\par}
    \date{8/29/2019}


\begin{document}
\maketitle

\hypertarget{multiple-testing}{%
\subsection{Multiple Testing}\label{multiple-testing}}

A simulation dataset where null is true for all features

\begin{Shaded}
\begin{Highlighting}[]
\KeywordTok{set.seed}\NormalTok{(}\DecValTok{1}\NormalTok{)}
\NormalTok{population <-}\StringTok{ }\KeywordTok{unlist}\NormalTok{(}\KeywordTok{read.csv}\NormalTok{(}\StringTok{"femaleControlsPopulation.csv"}\NormalTok{, }\DataTypeTok{stringsAsFactors =}\NormalTok{ F))}
\NormalTok{alpha <-}\StringTok{ }\FloatTok{0.05}
\NormalTok{N <-}\StringTok{ }\DecValTok{12}
\NormalTok{m <-}\StringTok{ }\DecValTok{10000}

\NormalTok{pvals <-}\StringTok{ }\KeywordTok{replicate}\NormalTok{(m, \{}
\NormalTok{  control <-}\StringTok{ }\KeywordTok{sample}\NormalTok{(population, N)}
\NormalTok{  treatment <-}\StringTok{ }\KeywordTok{sample}\NormalTok{(population, N)}
  \KeywordTok{t.test}\NormalTok{(control, treatment)}\OperatorTok{$}\NormalTok{p.value}
\NormalTok{\})}
\end{Highlighting}
\end{Shaded}

As we know no feature should be significantly different, However\ldots{}

\begin{Shaded}
\begin{Highlighting}[]
\KeywordTok{sum}\NormalTok{(pvals }\OperatorTok{<}\StringTok{ }\FloatTok{0.05}\NormalTok{)}
\end{Highlighting}
\end{Shaded}

\begin{verbatim}
## [1] 462
\end{verbatim}

462 False positives. Type-I error

Another dataset where 10\% of the features are true positives, and the
differences are 3

\begin{Shaded}
\begin{Highlighting}[]
\NormalTok{alpha <-}\StringTok{ }\FloatTok{0.05}
\NormalTok{N <-}\StringTok{ }\DecValTok{12}
\NormalTok{m <-}\StringTok{ }\DecValTok{10000}
\NormalTok{p0 <-}\StringTok{ }\FloatTok{0.9} \CommentTok{#10% positive}
\NormalTok{m0 <-}\StringTok{ }\NormalTok{m }\OperatorTok{*}\StringTok{ }\NormalTok{p0}
\NormalTok{m1 <-}\StringTok{ }\NormalTok{m }\OperatorTok{-}\StringTok{ }\NormalTok{m0}
\NormalTok{nullHypothesis <-}\StringTok{ }\KeywordTok{c}\NormalTok{(}\KeywordTok{rep}\NormalTok{(}\OtherTok{TRUE}\NormalTok{, m0), }\KeywordTok{rep}\NormalTok{(}\OtherTok{FALSE}\NormalTok{, m1))}
\NormalTok{delta <-}\StringTok{ }\DecValTok{3}

\KeywordTok{set.seed}\NormalTok{(}\DecValTok{1}\NormalTok{)}
\NormalTok{calls <-}\StringTok{ }\KeywordTok{sapply}\NormalTok{(}\DecValTok{1}\OperatorTok{:}\NormalTok{m, }\ControlFlowTok{function}\NormalTok{(i)\{}
\NormalTok{  control <-}\StringTok{ }\KeywordTok{sample}\NormalTok{(population, N)}
\NormalTok{  treatment <-}\StringTok{ }\KeywordTok{sample}\NormalTok{(population, N)}
  \ControlFlowTok{if}\NormalTok{ (}\OperatorTok{!}\NormalTok{nullHypothesis[i])\{}
\NormalTok{    treatment <-}\StringTok{ }\NormalTok{treatment }\OperatorTok{+}\StringTok{ }\NormalTok{delta}
\NormalTok{  \}}
  \KeywordTok{ifelse}\NormalTok{(}\KeywordTok{t.test}\NormalTok{(treatment, control)}\OperatorTok{$}\NormalTok{p.value }\OperatorTok{<}\StringTok{ }\NormalTok{alpha,}
         \StringTok{"Called significant"}\NormalTok{,}
         \StringTok{"Not called significant"}\NormalTok{)}
\NormalTok{\})}
\end{Highlighting}
\end{Shaded}

See false positives (Null is True but called significant) And false
negatives (Null is False but not called significant)

\begin{Shaded}
\begin{Highlighting}[]
\NormalTok{null_hypothesis <-}\StringTok{ }\KeywordTok{factor}\NormalTok{(nullHypothesis, }\DataTypeTok{levels =} \KeywordTok{c}\NormalTok{(}\StringTok{"TRUE"}\NormalTok{, }\StringTok{"FALSE"}\NormalTok{)) }\CommentTok{#Reorder}
\KeywordTok{table}\NormalTok{(null_hypothesis, calls)}
\end{Highlighting}
\end{Shaded}

\begin{verbatim}
##                calls
## null_hypothesis Called significant Not called significant
##           TRUE                 421                   8579
##           FALSE                520                    480
\end{verbatim}

Notice that false positives and false negatives are also random
variables, and can change in different simulations

\begin{Shaded}
\begin{Highlighting}[]
\NormalTok{B <-}\StringTok{ }\DecValTok{10} \CommentTok{#repeat 10 times}

\NormalTok{FPandFN <-}\StringTok{ }\KeywordTok{replicate}\NormalTok{(B, \{}
\NormalTok{  calls <-}\StringTok{ }\KeywordTok{sapply}\NormalTok{(}\DecValTok{1}\OperatorTok{:}\NormalTok{m, }\ControlFlowTok{function}\NormalTok{(i)\{}
\NormalTok{  control <-}\StringTok{ }\KeywordTok{sample}\NormalTok{(population, N)}
\NormalTok{  treatment <-}\StringTok{ }\KeywordTok{sample}\NormalTok{(population, N)}
  \ControlFlowTok{if}\NormalTok{ (}\OperatorTok{!}\NormalTok{nullHypothesis[i])\{}
\NormalTok{    treatment <-}\StringTok{ }\NormalTok{treatment }\OperatorTok{+}\StringTok{ }\NormalTok{delta}
\NormalTok{  \}}
  \KeywordTok{t.test}\NormalTok{(treatment, control)}\OperatorTok{$}\NormalTok{p.value }\OperatorTok{<}\StringTok{ }\NormalTok{alpha}
\NormalTok{  \})}
  \KeywordTok{cat}\NormalTok{(}\StringTok{"False positives = "}\NormalTok{, }\KeywordTok{sum}\NormalTok{(nullHypothesis }\OperatorTok{&}\StringTok{ }\NormalTok{calls), }
      \StringTok{"False negatives = "}\NormalTok{, }\KeywordTok{sum}\NormalTok{(}\OperatorTok{!}\NormalTok{nullHypothesis }\OperatorTok{&}\StringTok{ }\OperatorTok{!}\NormalTok{calls), }\StringTok{"}\CharTok{\textbackslash{}n}\StringTok{"}\NormalTok{)}
  \KeywordTok{c}\NormalTok{(}\KeywordTok{sum}\NormalTok{(nullHypothesis }\OperatorTok{&}\StringTok{ }\NormalTok{calls), }\KeywordTok{sum}\NormalTok{(}\OperatorTok{!}\NormalTok{nullHypothesis }\OperatorTok{&}\StringTok{ }\OperatorTok{!}\NormalTok{calls))}
\NormalTok{\})}
\end{Highlighting}
\end{Shaded}

\begin{verbatim}
## False positives =  410 False negatives =  436 
## False positives =  400 False negatives =  448 
## False positives =  366 False negatives =  454 
## False positives =  382 False negatives =  447 
## False positives =  372 False negatives =  495 
## False positives =  382 False negatives =  470 
## False positives =  381 False negatives =  461 
## False positives =  396 False negatives =  446 
## False positives =  380 False negatives =  450 
## False positives =  405 False negatives =  431
\end{verbatim}

\hypertarget{the-bonferroni-correction}{%
\subsection{The Bonferroni Correction}\label{the-bonferroni-correction}}

Too stringent, basically dividing alpha by number of samples In the
example below, 10\% of all the data are positives but only 2 are
detected\ldots{} High false negatives\ldots{}

\begin{Shaded}
\begin{Highlighting}[]
\KeywordTok{set.seed}\NormalTok{(}\DecValTok{1}\NormalTok{)}
\NormalTok{pvals <-}\StringTok{ }\KeywordTok{sapply}\NormalTok{(}\DecValTok{1}\OperatorTok{:}\NormalTok{m, }\ControlFlowTok{function}\NormalTok{(i)\{}
\NormalTok{  control <-}\StringTok{ }\KeywordTok{sample}\NormalTok{(population, N)}
\NormalTok{  treatment <-}\StringTok{ }\KeywordTok{sample}\NormalTok{(population, N)}
  \ControlFlowTok{if}\NormalTok{(}\OperatorTok{!}\NormalTok{nullHypothesis[i]) treatment <-}\StringTok{ }\NormalTok{treatment }\OperatorTok{+}\StringTok{ }\NormalTok{delta}
  \KeywordTok{t.test}\NormalTok{(control, treatment)}\OperatorTok{$}\NormalTok{p.value}
\NormalTok{\})}

\KeywordTok{sum}\NormalTok{(pvals }\OperatorTok{<}\StringTok{ }\NormalTok{alpha}\OperatorTok{/}\NormalTok{m)}
\end{Highlighting}
\end{Shaded}

\begin{verbatim}
## [1] 2
\end{verbatim}

\hypertarget{false-discovery-rate}{%
\subsection{False Discovery Rate}\label{false-discovery-rate}}


\end{document}
